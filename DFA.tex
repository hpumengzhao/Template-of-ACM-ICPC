%==============================常用宏包、环境==============================%
\documentclass[twocolumn,a4]{article}
\usepackage{xeCJK} % For Chinese characters
\usepackage{amsmath, amsthm}
\usepackage{listings,xcolor}
\usepackage{geometry} % 设置页边距
\usepackage{fontspec}
\usepackage{graphicx}
\usepackage{fancyhdr} % 自定义页眉页脚
\setsansfont{Consolas} % 设置英文字体
\setmonofont[Mapping={}]{Consolas} % 英文引号之类的正常显示,相当于设置英文字体
\geometry{left=1cm,right=1cm,top=2cm,bottom=0.5cm} % 页边距
\setlength{\columnsep}{30pt}
% \setlength\columnseprule{0.4pt} % 分割线
%==============================常用宏包、环境==============================%

%==============================页眉、页脚、代码格式设置==============================%
% 页眉、页脚设置
\pagestyle{fancy}
% \lhead{CUMTB}
\lhead{\CJKfamily{hei} Template of Team DFA}
\chead{}
% \rhead{Page \thepage}
\rhead{\CJKfamily{hei} 第 \thepage 页}
\lfoot{} 
\cfoot{}
\rfoot{}
\renewcommand{\headrulewidth}{0.4pt} 
\renewcommand{\footrulewidth}{0.4pt}

% 代码格式设置
\lstset{
    language    = c++,
    numbers     = left,
    numberstyle = \tiny,
    breaklines  = true,
    captionpos  = b,
    tabsize     = 4,
    frame       = shadowbox,
    columns     = fullflexible,
    commentstyle = \color[RGB]{0,128,0},
    keywordstyle = \color[RGB]{0,0,255},
    basicstyle   = \small\ttfamily,
    stringstyle  = \color[RGB]{148,0,209}\ttfamily,
    rulesepcolor = \color{red!20!green!20!blue!20},
    showstringspaces = false,
}
%==============================页眉、页脚、代码格式设置==============================%

%==============================标题和目录==============================%
\title{\CJKfamily{hei} \bfseries Template of Team DFA}
\author{HPU :  Codancer \& Dicer}
\renewcommand{\today}{\number\year 年 \number\month 月 \number\day 日}

\begin{document}\small
\begin{titlepage}
\maketitle
\end{titlepage}

\newpage
\pagestyle{empty}
\renewcommand{\contentsname}{目录}
\tableofcontents
\newpage\clearpage
\newpage
\pagestyle{fancy}
\setcounter{page}{1}   %new page
%==============================标题和目录==============================%

%==============================正文部分==============================%
\section{杂项}
\subsection{Head \&  快速读入}
\begin{lstlisting}
#include<bits/stdc++.h>
using namespace std;
const int N = 1e6+100;
const int mod = 1e9+7;
typedef long long ll;
const int INF = 0x3f3f3f3f;
const ll llINF = 0x3f3f3f3f3f3f3f3f;
#define rep(i,a,b) for(int i=(a);i<=(b);i++)
#define fep(i,a,b) for(int i=(a);i>=(b);i--)
inline bool read(ll &num) {
    char in;bool IsN=false;
    in=getchar();
    if(in==EOF) return false;
    while(in!='-'&&(in<'0'||in>'9')) in=getchar();
    if(in=='-'){ IsN=true;num=0;}
    else num=in-'0';
    while(in=getchar(),in>='0'&&in<='9'){
            num*=10,num+=in-'0';
    } 
    if(IsN) num=-num;
    return true;
}
\end{lstlisting}
\subsection{\_\_int128输入输出}
\begin{lstlisting}
void scan(__int128 &x)//输入{
    x = 0;
    int f = 1;
    char ch;
    if((ch = getchar()) == '-') f = -f;
    else x = x*10 + ch-'0';
    while((ch = getchar()) >= '0' && ch <= '9')
        x = x*10 + ch-'0';
    x *= f;
}
void print(__int128 x)//输出
{
    if(x < 0)
    {
        x = -x;
        putchar('-');
    }
     if(x > 9) print(x/10);
    putchar(x%10 + '0');
}
\end{lstlisting}
\subsection{O3优化}
\begin{lstlisting}
#pragma GCC optimize(3,"Ofast","inline")
\end{lstlisting}
\subsection{单调栈}
求第i个数作为最大值的区间[1,n]
\begin{lstlisting}
#include<bits/stdc++.h>

using namespace std;
const int N = 1e6+100;
int a[N];
int L[N],R[N];
int main(){
	int n;
	cin>>n;
	stack<int> sta;
	for(int i=1;i<=n;i++) scanf("%d",&a[i]);
	for(int i=1;i<=n;i++){
		while(sta.size()&&a[sta.top()]<=a[i]) sta.pop();
		if(sta.empty()) L[i]=1;
		else L[i]=sta.top()+1;
		sta.push(i);
	}
	while(sta.size()) sta.pop();
	for(int i=n;i>=1;i--){
		while(sta.size()&&a[sta.top()]<=a[i]) sta.pop();
		if(sta.empty()) R[i]=n;
		else R[i]=sta.top()-1;
		sta.push(i);
	}
\end{lstlisting}
\subsection{打印LIS}
\begin{lstlisting}
#include<bits/stdc++.h>

using namespace std;
const int N = 60000;
const int INF = 0X3f3f3f3f;
int dp[N],fa[N],a[N],b[N],order[N],n,pos[N];
bool vis[N],ok[N];
int solve(){//求b的LIS
	int cnt=0;
	for(int i=1;i<=n;i++){
		if(!vis[i]) b[cnt++]=a[i];//b数组
	}
	memset(dp,INF,sizeof(dp));
	memset(ok,0,sizeof(ok));
	int lpos;
	pos[0]=-1;
	for(int i=0;i<cnt;i++){
		dp[lpos=(lower_bound(dp,dp+cnt,b[i])-dp)]=b[i];
		pos[lpos]=i;
		fa[i]=(lpos?pos[lpos-1]:-1);
	}
	cnt=lower_bound(dp,dp+cnt,INF)-dp;
	int i;
	for(i=pos[cnt-1];~fa[i];i=fa[i]){
		 ok[b[i]]=1;//说明b[i]在LIS内
	}
	ok[b[i]]=1;
	return cnt;
}
\end{lstlisting}
\section{图论}
\subsection{Dinic最大流}
\begin{lstlisting}
二分图最大匹配
#include<bits/stdc++.h>

using namespace std;
const int N = 3000;
const int INF = 0x3f3f3f3f;
int S,T,n,m,w[N],dep[N],head[N],to[N],num=1,sum=0,x,nxt[N];
bool vis[N];
void add(int u,int v,int ww){
    num++;
    to[num]=v;nxt[num]=head[u];w[num]=ww;head[u]=num;
    num++;
    to[num]=u;nxt[num]=head[v];w[num]=0;head[v]=num;
}
queue<int> q;
bool bfs(){
    while(!q.empty()) q.pop();
    memset(vis,0,sizeof(vis));
    while(!q.empty()) q.pop();
    vis[S]=1;q.push(S);dep[S]=1;
    while(!q.empty()){
        int u=q.front();q.pop();
        for(int i=head[u];i;i=nxt[i]){
            int v=to[i];
            if(vis[v]||w[i]<=0) continue;
            dep[v]=dep[u]+1;
            vis[v]=1;q.push(v);
        }
    }
    return vis[T];
}
int dfs(int u,int d){
    if(u==T||d==0){
        return d;
    }
    int ret=0;
    for(int i=head[u];i;i=nxt[i]){
        int v=to[i];
        if(dep[v]!=dep[u]+1||w[i]<=0) continue;
        int flow=dfs(v,min(d,w[i]));
        d-=flow;ret+=flow;
        w[i]-=flow;w[i^1]+=flow;
        if(d==0) break;
    }
    if(ret==0) dep[u]=-1;
    return ret;
}
int main(){
    scanf("%d %d",&n,&m);
    S=0;T=n+1;
    int u,v;
    for(int i=1;i<=m;i++) add(S,i,1);
    for(int i=m+1;i<=n;i++) add(i,T,1);
    while(~scanf("%d %d",&u,&v)){
        add(u,v,1);
    }
    while(bfs()) sum+=dfs(S,INF);
    printf("%d\n",sum);
    return 0;
}

\end{lstlisting}
\subsection{倍增求解LCA}
\begin{lstlisting}
#include<bits/stdc++.h>

using namespace std;
const int N = 1e6+100;
vector<int> G[N];
long long bit[30];
int f[N][30];
int depth[N];
void init(){
	bit[0]=1;
	for(int i=1;i<=29;i++) bit[i]=(bit[i-1]<<1);
}
void dfs(int u,int par){
	depth[u]=depth[par]+1;
	f[u][0]=par;
	for(int i=1;bit[i]<=depth[u];i++) f[u][i]=f[f[u][i-1]][i-1];
	for(int i=0;i<(int)G[u].size();i++){
		int v=G[u][i];
		if(v!=par) dfs(v,u);
	}
}
int lca(int x,int y){
	if(depth[x]<depth[y]) swap(x,y);
	for(int i=29;i>=0;i--){
		if(depth[x]-depth[y]>=bit[i]){
			x=f[x][i];
		}
	}
	if(x==y) return x;
	for(int i=29;i>=0;i--){
		if(depth[x]>=(1<<i)&&f[x][i]!=f[y][i]){
			x=f[x][i];
			y=f[y][i];
		}
	}
	return f[x][0];
}
\end{lstlisting}
\subsection{有向图最小环}
\begin{lstlisting}
rep(k,1,n){
    rep(i,1,k-1){
        rep(j,1,i-1){
            ans=min(ans,dis[i][j]+val[i][k]+val[k][j]);//val代表边权
        }
    }
    rep(i,1,n){
        rep(j,1,n){
            dis[i][j]=min(dis[i][j],dis[i][k]+dis[k][j]);
        }
    }
}
\end{lstlisting}
\subsection{Tarjan}
\subsubsection{缩点求SCC}
\begin{lstlisting}
#include<bits/stdc++.h>

using namespace std;
const int N = 1e5+100;
typedef long long ll;
const int INF = 0x3f3f3f3f;
int n,m,scc,index;
vector<int> G[N];
ll w[N],low[N],dfn[N],minn[N],color[N],id[N];
bool is_instack[N];stack<int> sta;
//不要忘记初始化!!!!
void init(){
	scc=index=0;
	memset(low,0,sizeof(low));
	memset(dfn,0,sizeof(dfn));
	memset(color,0,sizeof(color));
	memset(minn,INF,sizeof(minn));
	memset(is_instack,0,sizeof(is_instack));
	for(int i=1;i<=n;i++) G[i].clear();
	while(!sta.empty()) sta.pop();
}
void Tarjan(int u){
	low[u]=dfn[u]=++index;
	sta.push(u);is_instack[u]=1;
	for(auto v:G[u]){
		if(!dfn[v]){
			Tarjan(v);
			low[u]=min(low[u],low[v]);
		}
		else if(is_instack[v]){
			low[u]=min(low[u],dfn[v]);
		}
	}
	if(low[u]==dfn[u]){
		++scc;
		while(1){
			int temp=sta.top();
			color[temp]=scc;
			minn[scc]=min(minn[scc],w[temp]);
			is_instack[temp]=0;
			sta.pop();
			if(temp==u) break;
		}
	}
}

//main函数Tarjan用法
main:
init();
for(int i=1;i<=n;i++){
	if(!dfn[i]) Tarjan(i);
}
\end{lstlisting}
\subsubsection{求割点}
割点去掉后各联通快大小
\begin{lstlisting}
#include<bits/stdc++.h>

using namespace std;
const int N = 1e5+100;
typedef long long ll;
int dfn[N],low[N];
ll n,m,siz[N];
int idx;
vector<ll> G[N],GD[N];
bool jud[N];//是否为割点
void Tarjan(int u,int fa){
    dfn[u]=low[u]=++idx;
    siz[u]=1;
    int allsiz=0;
    for(int v:G[u]){
        if(!dfn[v]){//没有访问过
            Tarjan(v,u);
            siz[u]+=siz[v];
            low[u]=min(low[u],low[v]);
            if(low[v]>=dfn[u]){//u为割点
                jud[u]=1;
                allsiz+=siz[v];
                GD[u].push_back(siz[v]);
            }
        }
        else if(v!=fa){
            low[u]=min(low[u],low[v]);
        }
    }
    if(jud[u]&&n-allsiz-1){
        GD[u].push_back(n-allsiz-1);
    }
}
int main(){
    idx=0;
    cin>>n>>m;
    int u,v;
    for(int i=1;i<=m;i++){
        cin>>u>>v;
        G[u].push_back(v);
        G[v].push_back(u);
    }
    Tarjan(1,0);
    for(int i=1;i<=n;i++){
        if(!jud[i]){
            cout<<2*(n-1)<<endl;
        }else{
            ll ans=n*(n-1);
            ll now=0;
            for(ll v:GD[i]){
                now+=v*(v-1);
            }
            cout<<ans-now<<endl;
        }
    }
    return 0;
}
\end{lstlisting}
\subsubsection{无向图点双连通}
\begin{lstlisting}
#include<bits/stdc++.h>

using namespace std;
const int N = 2e5+100;
int n,m;
int bcc_cnt;//bcc个数
int dfs_clock;
int pre[N];
bool is_cut[N];//判断是否是割点
int bccno[N];//第i个点是属于哪一个双连通分量
vector<int> G[N],bcc[N];

struct edge{
    int u,v;
    edge(int u,int v):u(u),v(v){}
};
stack<edge> s;
int Tarjan(int u,int fa){
    int lowu=pre[u]=++dfs_clock;
    int child=0;
    for(int v:G[u]){
        edge e=edge(u,v);
        if(!pre[v]){
            s.push(e);
            child++;
            int lowv=Tarjan(v,u);
            lowu=min(lowv,lowu);
            if(lowv>=pre[u]){//找到了割点
                is_cut[u]=1;
                bcc_cnt++;
                bcc[bcc_cnt].clear();
                while(1){
                    edge x=s.top();s.pop();
                    if(bccno[x.u]!=bcc_cnt){
                        bcc[bcc_cnt].push_back(x.u);
                        bccno[x.u]=bcc_cnt;
                    }
                    if(bccno[x.v]!=bcc_cnt){
                        bcc[bcc_cnt].push_back(x.v);
                        bccno[x.v]=bcc_cnt;
                    }
                    if(x.u==u&&x.v==v) break;
                }
            }
        }else if(pre[v]<pre[u]&&v!=fa){
            s.push(e);
            lowu=min(lowu,pre[v]);
        }
    }
    if(fa<0&&child==1) is_cut[u]=0;
    return lowu;
}
void find_bcc(int n){
    for(int i=0;i<=n;i++) pre[i]=is_cut[i]=bccno[i]=0;
    for(int i=1;i<=n;i++){
        if(!pre[i]) Tarjan(i,-1);
    }
}
int main(){
    int cnt=0;
    cin>>n>>m;
    int u,v;
    for(int i=0;i<m;i++){
        cin>>u>>v;
        G[u].push_back(v);
        G[v].push_back(u);
    }
    find_bcc(n);
    cout<<bcc_cnt<<endl;
    for(int i=1;i<=bcc_cnt;i++){
        cout<<"BCC "<<i<<endl;
        for(int v:bcc[i]) cout<<v<<' ';
        cout<<endl;
    }
    return 0;
}
\end{lstlisting}
\subsubsection{无向图边双连通}
该程序是判断两点之间是否有两条不相交的路径
\begin{lstlisting}
#include <bits/stdc++.h>
using namespace std;
const int N=5e4+5,M=1e5+5;
int n,m,vis[M<<1],ans;
int cnt=1,head[N],u[M],v[M];
int now,col,dfn[N],low[N],color[N],q,x,y;
stack<int> sta;
struct edge{int next,to;}e[M<<1];
inline void add(int u,int v){
    cnt++;
    e[cnt].next=head[u];
    e[cnt].to=v;
    head[u]=cnt;
    cnt++;
    e[cnt].next=head[v];
    e[cnt].to=u;
    head[v]=cnt;
}
inline void tarjan(int u)
{
    dfn[u]=low[u]=++now;    
    sta.push(u);
    for (int i=head[u];i; i=e[i].next){
        if(!vis[i]){
            vis[i]=vis[i^1]=1;
            if (!dfn[e[i].to]){
                tarjan(e[i].to);    
                low[u]=min(low[u],low[e[i].to]);
            }
            else low[u]=min(low[u],dfn[e[i].to]);
        }
    }
    if (low[u]==dfn[u]){
        color[u]=++col; 
        while (1){
            int now=sta.top();sta.pop();
            color[now]=col;
            if(now==u) break;
        }
    }
}

int main(){
    memset(head,0,sizeof(head));
    memset(dfn,0,sizeof(head));
    scanf("%d%d",&n,&m);
    for (int i=1; i<=m; ++i){
         scanf("%d%d",&u[i],&v[i]);
         add(u[i],v[i]);
    }
    for (int i=1; i<=n; ++i){
         if (!dfn[i]) tarjan(i);
    }
    scanf("%d",&q);
    while(q--){
        scanf("%d %d",&x,&y);
        if(color[x]!=color[y]){
            puts("No");
        }else{
            puts("Yes");
        }
    }
    return 0;   
}
\end{lstlisting}
\subsection{求无向连通图的第K大联通子图}
\begin{lstlisting}
#include<bits/stdc++.h>
 
using namespace std;
bitset<105> bs[105];
long long w[105];
char maze[105][105];
struct node{
    bitset<105> cl;
    long long w;
    int last;
};
bool operator<(node a,node b){
    return a.w>b.w;
}
priority_queue<node> q;
int n,k;
long long bfs(){
    node now;
    now.cl.reset();
    now.w=0;
    now.last=0;
 
    q.push(now);
    while(!q.empty()){
        node rt=q.top();q.pop();
        if(--k==0) return rt.w;
 
        for(int i=rt.last+1;i<=n;i++){
            if(((bs[i]&rt.cl).count()==rt.cl.count())){
                node pt=rt;
                pt.cl[i]=1;
                pt.last=i;
                pt.w+=w[i];
                q.push(pt);
            }
        }
    }
    return -1;
}
int main(){
    cin>>n>>k;
    for(int i=1;i<=n;i++) cin>>w[i];
    for(int i=1;i<=n;i++){
        for(int j=1;j<=n;j++){
            cin>>maze[i][j];
            if(maze[i][j]=='1'){
                bs[i][j]=1;
            }
        }
    }
    cout<<bfs()<<endl;
    return 0;
}

\end{lstlisting}

\subsection{分层(K)图最短路}
\begin{lstlisting}
#include <bits/stdc++.h>
using namespace std;
typedef long long ll;
typedef pair<int ,int> pii;
const int INF = 0x3f3f3f3f;
const int N = 5e6+10;
const int M = 5e6+10;
  
struct EDGE{
    int next;
    int to;
    int w;
}edge[M];
  
int n,m,k,s,e,cnt = 1;
int head[N],dis[N];
bool inq[N];
  
void add(int u, int v, int w){
    edge[cnt].next = head[u];
    edge[cnt].to = v;
    edge[cnt].w  = w;
    head[u] = cnt++;
}
  
struct NODE{
    int id,dist;
}q,p;
bool operator < (NODE a, NODE b){
    return a.dist > b.dist;
}
void Dijkstra(){
    memset(dis, INF, sizeof dis);
    memset(inq, 0, sizeof inq);
    priority_queue<NODE> que;
    p.id = s; p.dist = 0;
    dis[s] = 0; que.push(p);
    while(!que.empty()){
        q = que.top(); que.pop();
        if(inq[q.id])   continue;
        inq[q.id] = true;
        for(int i=head[q.id]; ~i; i=edge[i].next){
            int u = edge[i].to;
            if(dis[u] > q.dist + edge[i].w){
                dis[u] = q.dist + edge[i].w;
                p.id = u;
                p.dist = dis[u];
                que.push(p);
            }
        }
    }
    int ans = INF;
    for(int i=0; i<=k; ++i)  ans = min(ans, dis[e + i*n]);
    printf("%d\n", ans);
}
int main(int argc, char const *argv[])
{
    int u, v, w;
    memset(head, -1, sizeof head);
    scanf("%d %d %d %d %d",&n, &m, &s, &e, &k);
    for(int i=1;i<=m;++i){
        scanf("%d %d %d",&u, &v, &w);
        for(int j=0; j<=k; ++j){
            add(u + j*n, v + j*n, w);
            add(v + j*n, u + j*n, w);
            if(j != k){
                add(u + j*n, v + (j+1)*n, 0);
                add(v + j*n, u + (j+1)*n, 0);
            }
        }
    }
    Dijkstra();
    return 0;
}
\end{lstlisting}
\subsection{点分治}
\begin{lstlisting}
typedef long long ll;
const int MOD = 1e9+7;
const int MAXN = 1e4 + 7;

int n, root, size, tot = 0;
int son[MAXN], f[MAXN], head[MAXN];
int dep[MAXN]; bool vis[MAXN];
struct node{
    int u, w, nxt;
}; vector<node> E;
void add(int u, int v, int w) {
    E.push_back(node{v, w, head[u]});
    head[u] = tot++;
}
void init() {
    memset(head, -1, sizeof head);
    memset(vis, 0, sizeof vis);
    E.clear(); tot = 0;
}
void get_rt(int x, int fa = 0) {
    son[x] = 1; f[x] = 0;
    for(int j = head[x]; ~j; j = E[j].nxt) {
        int u = E[j].u, w = E[j].w;
        if(vis[u] || u == fa)   continue;
        get_rt(u, x);
        son[x] += son[u];
        f[x] = max(f[x], son[u]);
    }
    f[x] = max(f[x], size - son[x]);
    if(f[x] < f[root]) root = x;
}

void get_dep(int x, int fa) {
    //得到每个节点到根节点的距离
    for(int j = head[x]; ~j; j = E[j].nxt) {
        int u = E[j].u, w = E[j].w;
        if(vis[u] || u == fa)   continue;
        dep[u] = dep[x] + w;
        get_dep(u, x);
    }
}
void calc(int x, int op) {
    get_dep(x, 0);
    //得到深度,处理答案
    //updata ans
}
void solve(int x) {
    dep[x] = 0; calc(x, 1); vis[x] = 1;
    for(int j = head[x]; ~j; j = E[j].nxt) {
        int u = E[j].u, w = E[j].w;
        if(vis[u])  continue;
        dep[u] = w; calc(u, -1);
        root = 0; size = son[u];
        get_rt(u);
        solve(root);
    }
}
int main(int argc, char const *argv[])
{
    while(~scanf("%d", &n)) {
        init();
        int u, v, w;
        rep(i, 1, n-1) {
            scanf("%d %d %d", &u, &v, &w);
            add(u, v, w); add(v, u, w);
        }
        root = 0; f[0] = size = n;
        get_rt(1, 0);
        //init ans
        solve(root);
        //print ans
    }
    return 0;
}

\end{lstlisting}
\section{数学}
\subsection{整除分块}
计算$\sum_{i=1}^{n} \lfloor\frac{n}{i}\rfloor$
\begin{lstlisting}
for(int l=1,r;l<=n;l=r+1)
{
    r=n/(n/l);
    ans+=(r-l+1)*(n/l);
}

\end{lstlisting}
\subsection{SG函数打表}
\begin{lstlisting}
int op[110],sg[11000];
int k,N;
vector<int > s;
void getSG(){
    sg[0] = 0;
    for(int i=1;i<=N;++i){
        s.clear();
        for(int j=1;i>=op[j] && j<=k;++j)
            s.push_back(sg[i - op[j]]);
        for(int j=0;;++j){
            if(count(s.begin(),s.end(),j) == 0){
                sg[i] = j;
                break;
            }
        }
    }
}
\end{lstlisting}
\subsection{线性素数+莫比乌斯函数打表}
\begin{lstlisting}
int miu[MAXN+10], check[MAXN+10], prime[MAXN+10];
void Mobius()
{
    memset(check,false,sizeof(check));
    miu[1] = 1;
    int tot = 0;
    for(int i = 2; i <= MAXN; i++)
    {
        if( !check[i] )
        {
            prime[tot++] = i;
            miu[i] = -1;
        }
        for(int j = 0; j < tot; j++)
        {
            if(i * prime[j] > MAXN) break;
            check[i * prime[j]] = true;
            if( i % prime[j] == 0)
            {
                miu[i * prime[j]] = 0;
                break;
            }
            else
            {
                miu[i * prime[j]] = -miu[i];
            }
        }
    }
}
\end{lstlisting}
\subsection{高斯消元}
\begin{lstlisting}
//计算n*(n+1)矩阵的解,ans[i]即为所求
#include<bits/stdc++.h>

using namespace std;
const double eps = 1e-7;
int n;
double a[110][110],ans[110];
void Gauss(int n){
    for(int i=1;i<=n;i++){
        int r=i;
        for(int j=i+1;j<=n;j++){
            if(fabs(a[r][i])<fabs(a[j][i])){
                r=j;
            }
        }
        if(fabs(a[r][i])<eps) return ;
        if(i!=r)swap(a[i],a[r]);
        double div=a[i][i];
        for(int j=i;j<=n+1;j++){
            a[i][j]/=div;
        }
        for(int j=i+1;j<=n;j++){
            div=a[j][i];
            for(int k=i;k<=n+1;k++){
                a[j][k]-=a[i][k]*div;
            }
        }
    }
    ans[n]=a[n][n+1];
    for(int i=n-1;i>=1;i--){
        ans[i]=a[i][n+1];
        for(int j=i+1;j<=n;j++){
            ans[i]-=(a[i][j]*ans[j]);
        }
    }//回带操作
}
\end{lstlisting}
\subsection{Lucas定理求C(n,m)\% P}
\begin{lstlisting}
typedef long long LL;

LL mod;

inline LL pow(LL a, LL b)//快速幂是为了求逆元
{
    LL ans = 1;
    for(; b; b >>= 1,a = a * a % mod)
        if(b & 1)
            ans = ans * a % mod;
    return ans;
}

LL farc[1000005];

inline void prepare(LL a)
{
    farc[0]=1;
    for(LL i = 1; i <= a; ++i)
        farc[i]=farc[i-1]*i%mod;
}

inline LL Csmall(LL m, LL n) // C(m,n) = (n!)/(m!*(n-m)!)
{
    if(n < m)
        return 0;
    return farc[n] * pow(farc[m], mod-2) % mod * pow(farc[n-m], mod-2) % mod; // 费马小定理求逆元
}

inline LL C(LL m, LL n)
{
    if(n < m)
        return 0;
    if(!n)
        return 1;//Lucas的边界条件
    return C(m/mod, n/mod) % mod * Csmall(m%mod, n%mod) % mod; // 上面证明的Lucas定理
}
\end{lstlisting}
\subsection{大数质因子分解\&大素数检测}
\begin{lstlisting}
ll Abs( ll a ){ return a<0?-a:a; }
ll Min( ll a , ll b ){ return a<b?a:b; }
ll Max( ll a , ll b ){ return a>b?a:b; }
ll Gcd( ll a , ll b ){ return b==0?a:Gcd( b , a%b ); }
ll arr[5] = { 2,3,5,233,331 };
ll Qmul( ll a , ll b , ll mod )
{
    ll res = 0;
    while ( b )
    {
        if ( b&1 )
            res = ( res+a )%mod;
        a = ( a+a )%mod;
        b = b>>1;
    }
    return res;
}
ll Qpow( ll a , ll b , ll mod )
{
    ll res = 1;
    while ( b )
    {
        if ( b&1 )
            res = Qmul( res , a , mod );
        a = Qmul( a , a , mod );
        b = b>>1;
    }
    return res;
}
bool Miller_Rabin( ll n )
{
    if ( n==2 ) return true;
    if ( n <2||n%2==0 ) return false;
    ll m = n-1,k = 0;
    while ( m%2==0 ) k++,m>>=1;
    for ( int I=0 ; I<5 ; I++ )
    {
        ll a = arr[I]%(n-1)+1;
        ll x = Qpow( a , m , n );
        for ( int j=1 ; j<=k ; j++ )
        {
            ll y = Qmul( x , x , n );
            if ( y==1&&x!=1&&x!=n-1 )
                return false;
            x = y;
        }
        if ( x!=1 ) return false;
    }
    return true;
}
ll fac[110], tol = 0;
ll Pollard_rho( ll x , ll c )
{
    ll I=1,k=2;
    ll x0 = rand()%x;
    ll y0 = x0;
    while ( 1 )
    {
        I++;
        x0 = ( Qmul( x0 , x0 , x )+c )%x;
        ll d0 = Gcd( Abs( y0-x0 ) , x );
        if ( d0!=1&&d0!=x ) return d0;
        if ( y0==x0 ) return x;
        if ( I == k ) { y0=x0; k+=k; }
    }
}
void Findfac( ll n )
{
    if ( Miller_Rabin( n ) )
    {
        fac[tol++] = n;
        return;
    }
    ll p = n;
    while ( p>=n )
        p = Pollard_rho( p , rand()%(n-1)+1 );
    Findfac( p );
    Findfac( n/p );
}
ll exgcd(ll a,ll b,ll &x,ll &y)
{
	if(b==0)
	{
		x=1,y=0;
		return a;
	}
	ll g=exgcd(b,a%b,x,y);
	ll tmp=x;x=y;y=tmp-a/b*y;
	return g;
}

\end{lstlisting}
\subsection{python通用中国剩余定理}
\begin{lstlisting}
"""
n 方程个数
a1 r1: x = a1 (mod r1)
flag 是否有解
""" 
def egcd(a, b):
    if 0 == b:
        return 1, 0, a
    x, y, q = egcd(b, a % b)
    x, y = y, (x - a // b * y)
    return x, y, q
n = int(input().split)
flag = False
a1, r1 = map(int, input().split())
for _ in range(n-1):
    a2, r2 = map(int, input().split())
    R = r2-r1
    x, y, d = egcd(a1, a2)
    tmp = a2//d
    if R%d != 0:
        flag = True
    r1=((x*R//d)%tmp+tmp)%tmp*a1+r1
    a1=a1*(a2//d)
lcm = a1
ans = (r1%lcm+lcm)%lcm
\end{lstlisting}
\subsection{在线求组合数}
\begin{lstlisting}
void init(){
    fact[0]=inv[1]=factinv[0]=inv[0]=fact[1]=factinv[1]=1;
    for(int i=2;i<=MAXN;i++){
        fact[i]=(fact[i-1]%mod*i%mod)%mod;
        inv[i]=(mod-mod/i)*inv[mod%i]%mod;
        factinv[i]=factinv[i-1]*inv[i]%mod;
    }
}
ll c(ll n,ll m){
    return fact[n]*factinv[m]%mod*factinv[n-m]%mod;
}
\end{lstlisting}
\subsection{拉格朗日插值}
\subsubsection{连续情况}
以计算$\sum_{i=1}^{n}{i^k}$为例
\begin{lstlisting}
#include<bits/stdc++.h>

using namespace std;
const int N = 1e6+100;
typedef long long ll;
const ll mod = 1e9+7;
ll p[N],x[N],s1[N],s2[N],ifac[N];
ll qpow(ll a,ll b){
	ll ans=1;
	while(b){
		if(b&1) ans=(ans%mod*a%mod)%mod;
		a=(a%mod*a%mod)%mod;
		b>>=1;
	}
	return (ans%mod+mod)%mod;
}

//拉格朗日插值,n项,每个点的坐标为(x_i,y_i),求第xi项的值,保证x是连续的一段 
ll lagrange(ll n, ll *x, ll *y, ll xi) {
    ll ans = 0;
    s1[0] = (xi-x[0])%mod, s2[n+1] = 1;
    for (ll i = 1; i <= n; i++) s1[i] = 1ll*s1[i-1]*(xi-x[i])%mod;
    for (ll i = n; i >= 0; i--) s2[i] = 1ll*s2[i+1]*(xi-x[i])%mod;
    ifac[0] = ifac[1] = 1;
    for (ll i = 2; i <= n; i++) ifac[i] = -1ll*mod/i*ifac[mod%i]%mod;
    for (ll i = 2; i <= n; i++) ifac[i] = 1ll*ifac[i]*ifac[i-1]%mod;
    for (ll i = 0; i <= n; i++)
        (ans += 1ll*y[i]*(i == 0 ? 1 : s1[i-1])%mod*s2[i+1]%mod
            *ifac[i]%mod*(((n-i)&1) ? -1 : 1)*ifac[n-i]%mod) %= mod;
    return (ans+mod)%mod;
}
int main(){
	ll n,k;
	cin>>n>>k;
	if(k==0){
		cout<<n<<endl;
		return 0;
	}
	p[0]=0;
	for(ll i=1;i<=k+2;i++) p[i]=(p[i-1]%mod+qpow(i,k))%mod;
	for(ll i=1;i<=k+2;i++) x[i]=i;
	if(n<=k+2){
		cout<<p[n]<<endl;
	}
	else{
		cout<<lagrange(k+2,x,p,n)<<endl;
	}
	return 0;
}
\end{lstlisting}
\subsubsection{非连续情况}
\begin{lstlisting}
#include<bits/stdc++.h>

using namespace std;
const int N = 1e6+100;
typedef long long ll;
const ll mod = 998244353;
struct point{
	ll x,y;
}p[N];
int n,k;
ll qpow(ll a,ll b,ll mod){
	ll ans=1;
	while(b){
		if(b&1){
			ans=(ans%mod*a%mod)%mod;
		}
		a=(a%mod*a%mod)%mod;
		b>>=1;
	}
	return ans%mod;
}
ll Lagrange(int k){
	ll ans=0;
	for(int j=1;j<=n;j++){//
		ll base1=1;
		ll base2=1;
		for(int i=1;i<=n;i++){//lj(k)基函数 
			if(j==i) continue;
			base1=(base1%mod*((k-p[i].x)%mod+mod)%mod)%mod;
			base2=(base2%mod*((p[j].x-p[i].x)%mod+mod)%mod)%mod;
		}
		ans=(ans%mod+(p[j].y%mod*base1%mod*qpow(base2,mod-2,mod)%mod)%mod)%mod;
	}
	return ans;
} 
int main(){
	cin>>n>>k;
	for(int i=1;i<=n;i++) cin>>p[i].x>>p[i].y;
	cout<<Lagrange(k)<<endl;
	return 0;
}


\end{lstlisting}
\subsection{辛普森自适应积分}
\begin{lstlisting}
#include<cstdio>
#include<cmath>
double a, b, c, d, L, R;
double F(double x) {
    return (c * x + d) / (a * x + b);
}
double sim(double l, double r) {
    return (F(l) + F(r) + 4 * F((l + r) / 2)) * (r - l) / 6;
}
double asr(double L, double R, double eps, double ans) {
    double mid = (L + R) / 2;
    double LL = sim(L, mid), RR = sim(mid, R);
    if(fabs(LL + RR - ans) < eps) return LL + RR;
    else return asr(L, mid, eps / 2, sim(L, mid)) + asr(mid, R, eps / 2, sim(mid, R));
}
main() {
    #ifdef WIN32
    freopen("a.in", "r", stdin);
    #endif
    scanf("%lf %lf %lf %lf %lf %lf", &a, &b, &c, &d, &L, &R);
    printf("%lf", asr(L, R, 1e-6, sim(L, R)));
\end{lstlisting}
\subsection{欧拉函数}
比n小的与n互质的数的个数
\subsubsection{在线}
\begin{lstlisting}

int euler(int n)//返回euler(n)
{
     int i;
     int res = n,a = n;
     for(i = 2;i*i <= a; ++i)
     {
         if(a%i == 0)
         {
             res -= res/i; //p(n) = (p - p/p1)(1 - 1/p2)......
             while(a%i == 0) a/=i;
         }
     }
     if(a > 1) res -= res/a;//存在大于sqrt(a)的质因子
     return res;
}
\end{lstlisting}
\subsubsection{打表}
\begin{lstlisting}
void SE()//select euler//类似于素数筛选法
{
    int i,j;
    euler[1] = 1;
    for(i = 2;i < Max; ++i)  euler[i]=i;
    for(i = 2;i < Max; ++i)
    {
         if(euler[i] == i)//这里出现的肯定是素数
         {
           for(j = i; j < Max; j += i)//然后更新含有它的数
           {
              euler[j] = euler[j]/i*(i - 1); // n*(1 - 1/p1)....*(1 - 1/pk).先除后乘
           }
        }
    }
     //for (int i = 1; i <= 20; ++i) printf("%d ",euler[i]);
}
\end{lstlisting}
\subsection{欧拉降幂}
降幂公式:
$$
a^b  \%  p=
\begin{cases}
a^{b \% \phi(p)} \%p & gcd(a,p)=1\\
a^b \% p & gcd(a,p) \neq1 ,b < \phi(p)\\
a^{b \% \phi(p)+\phi(p)} \% p & gcd(a,p)\neq 1, \phi(p) \leq b
\end{cases}
$$

\begin{lstlisting}
#include<bits/stdc++.h>
using namespace std;
const int N = 2e6+100;
const int mod = 1e9+7;
typedef long long ll;
const int INF = 0x3f3f3f3f;
const ll llINF = 0x3f3f3f3f3f3f3f3f;
#define rep(i,a,b) for(int i=(a);i<=(b);i++)
#define fep(i,a,b) for(int i=(a);i>=(b);i--)
inline bool read(ll &num) {
	char in;bool IsN=false;
	in=getchar();
	if(in==EOF) return false;
	while(in!='-'&&(in<'0'||in>'9')) in=getchar();
	if(in=='-'){ IsN=true;num=0;}
	else num=in-'0';
	while(in=getchar(),in>='0'&&in<='9'){
			num*=10,num+=in-'0';
	} 
	if(IsN) num=-num;
	return true;
}
ll ph[N];
void init(){
	rep(i,1,N-10){
		ph[i]=i;
	}
	rep(i,2,N-10){
		if(ph[i]==i){
			for(int j=i;j<=N-10;j+=i){
				ph[j]=ph[j]/i*(i-1);
			}
		}
	}
}
ll qpow(ll a,ll b,ll mod){
	ll ans=1;
	while(b){
		if(b&1) ans=(ans%mod*a%mod)%mod;
		a=(a%mod*a%mod)%mod;
		b>>=1;
	}
	return ans%mod;
}
bool check(ll a,ll b,ll m){
	if(b==0) return 1>=ph[m];
	if(b==1) return a>=ph[m];
	ll ans=1;
	if(ans>=ph[m]) return 1;
	rep(i,1,b-1){
		rep(j,1,a){
			ans*=a;
			if(ans>=ph[m]) return 1;
		}
	}
	return 0;
}
ll  solve(ll a,ll b,ll m){
	if(m==1) return 0;
	if(b==0) return 1%m;
	if(b==1) return a%m;
	if(__gcd(a,m)==1){
		return qpow(a,solve(a,b-1,ph[m]),m);
	}
	else{
		if(check(a,b-1,m)){
			return qpow(a,solve(a,b-1,ph[m])+ph[m],m);
		}
		else return qpow(a,solve(a,b-1,m),m);
	}
}
ll T,a,b,m;
int main(){
	//freopen("1.in", "r", stdin);
	read(T);
	init();
	//cout<<ph[1000000]<<endl;
	while(T--){
		read(a);read(b);read(m);
		printf("%lld\n",solve(a,b,m)%m);
	}
	return 0;
}

\end{lstlisting}
\section{数据结构}
\subsection{线段树}
\subsubsection{区间修改区间查询}
\begin{lstlisting}
#include<bits/stdc++.h>
using namespace std;
typedef long long ll;
const ll maxn = 1000050;
ll ans[maxn<<1],a[maxn],mod;
ll add[maxn],mult[maxn];
inline void pushup(ll rt){
	ans[rt]=(ans[rt<<1]+ans[rt<<1|1])%mod;
}
void pushdown(ll rt,ll l,ll r){
	ll mid=(l+r)>>1;
	ans[rt<<1]=(ans[rt<<1]*mult[rt]+add[rt]*(mid-l+1))%mod;
	ans[rt<<1|1]=(ans[rt<<1|1]*mult[rt]+add[rt]*(r-mid))%mod;
	
	mult[rt<<1]=(mult[rt]*mult[rt<<1])%mod;
	mult[rt<<1|1]=(mult[rt]*mult[rt<<1|1])%mod;
	
	add[rt<<1]=(add[rt<<1]*mult[rt]+add[rt])%mod;
	add[rt<<1|1]=(add[rt<<1|1]*mult[rt]+add[rt])%mod;
	
	add[rt]=0;
	mult[rt]=1;
	return ;
}
inline void buildtree(ll rt,ll l,ll r){
	mult[rt]=1;
	add[rt]=0;
	if(l==r){
		ans[rt]=a[l];
		return ;
	}
	ll mid=(l+r)>>1;
	buildtree(rt<<1,l,mid);
	buildtree(rt<<1|1,mid+1,r);
	pushup(rt);
}
inline void update1(ll nl,ll nr,ll l,ll r,ll rt,ll k){
	if(nl<=l&&r<=nr){
		ans[rt]=(ans[rt]*k)%mod;
		add[rt]=(add[rt]*k)%mod;
		mult[rt]=(mult[rt]*k)%mod;
		return ;
	}
	pushdown(rt,l,r);
	ll mid=(l+r)>>1;
	if(nl<=mid){
		update1(nl,nr,l,mid,rt<<1,k);
	}
	if(nr>mid) update1(nl,nr,mid+1,r,rt<<1|1,k);
	pushup(rt);
}
inline void update2(ll nl,ll nr,ll l,ll r,ll rt,ll k){
	if(nl<=l&&nr>=r){
		add[rt]=(add[rt]+k)%mod;
		ans[rt]=(ans[rt]+k*(r-l+1))%mod;
		return ;
	}
	pushdown(rt,l,r);
	ll mid=(l+r)>>1;
	if(nl<=mid){
		update2(nl,nr,l,mid,rt<<1,k);
	}
	if(nr>mid) update2(nl,nr,mid+1,r,rt<<1|1,k);
	pushup(rt);
}
ll query(ll nl,ll nr,ll l,ll r,ll rt){
	ll res=0;
	if(nl<=l&&r<=nr){
		return ans[rt]%mod;
	}
	ll mid=(l+r)>>1;
	pushdown(rt,l,r);
	if(nl<=mid) res=(res%mod+query(nl,nr,l,mid,rt<<1))%mod;
	if(nr>mid) res=(res%mod+query(nl,nr,mid+1,r,rt<<1|1))%mod;
	return res;
}
int main(){
	ios::sync_with_stdio(0);
	cin.tie(0);
	cout.tie(0);
	ll n,m,op,x,y,k;
	cin>>n>>m>>mod;
	for(ll i=1;i<=n;i++) cin>>a[i];
	buildtree(1,1,n);
	while(m--){
		cin>>op;
		if(op==1){
			cin>>x>>y>>k;
			update1(x,y,1,n,1,k);
		}
		else if(op==2){
			cin>>x>>y>>k;
			update2(x,y,1,n,1,k);
		}
		else{
			cin>>x>>y;
			cout<<query(x,y,1,n,1)<<endl;
		}
	}
	return 0;
}
\end{lstlisting}
\subsection{主席树}
\subsubsection{区间第k小}
\begin{lstlisting}
#include<bits/stdc++.h>

using namespace std;
const int N = 1e5+100;
struct node{
	int l,r,num;
}T[N*30];
vector<int> v;
int n,m,a[N],t,cnt,roots[N];
int getid(int x){
	return lower_bound(v.begin(),v.end(),x)-v.begin()+1;
}
void update(int l,int r,int &x,int y,int pos){
	T[++cnt]=T[y];T[cnt].num++;x=cnt;
	if(l==r) return ;
	int mid=(l+r)>>1;
	if(pos<=mid) update(l,mid,T[x].l,T[y].l,pos);
	else update(mid+1,r,T[x].r,T[y].r,pos);
}
int query(int l,int r,int x,int y,int k){
	if(l==r) return l;
	int sum=T[T[x].l].num-T[T[y].l].num;
	int mid=(l+r)>>1;
	if(sum>=k) return query(l,mid,T[x].l,T[y].l,k);
	else return query(mid+1,r,T[x].r,T[y].r,k-sum);
}
int main(){
	scanf("%d",&t);
	while(t--){
		v.clear();
		cnt=0;
		scanf("%d %d",&n,&m);
		for(int i=1;i<=n;i++){
			scanf("%d",&a[i]);
			v.push_back(a[i]);
		}
		sort(v.begin(),v.end());
		v.erase(unique(v.begin(),v.end()),v.end());
		for(int i=1;i<=n;i++) update(1,n,roots[i],roots[i-1],getid(a[i]));
		while(m--){
			int l,r,k;
			scanf("%d %d %d",&l,&r,&k);
			printf("%d\n",v[query(1,n,roots[r],roots[l-1],k)-1]);
		}
	}
	return 0;
}
\end{lstlisting}
\subsubsection{区间内小于等于x的最大值}
\begin{lstlisting}
struct node
{
	ll sum,l,r;
}t[maxn*32];
int cnt;
void update(ll l,ll r,ll &x,ll y,ll pos){
	t[++cnt]=t[y];t[cnt].sum++;x=cnt;//复制节点并且更新
	if(l==r) return ;
	int mid=(l+r)>>1;
	if(mid>=pos) update(l,mid,t[x].l,t[y].l,pos);
	else update(mid+1,r,t[x].r,t[y].r,pos);
}
int query(int a,int b,int x,int l,int r)
{
	if(l==r)
	{
		if(l==x)
			return 0;
		else
			return l;
	}
	int mid=(l+r)>>1;
	int xx=t[t[b].l].sum-t[t[a].l].sum;
	int yy=t[t[b].r].sum-t[t[a].r].sum;
	int res=0;
	if(yy&&x>mid)
		res=query(t[a].r,t[b].r,x,mid+1,r);
	if(xx&&!res)
		res=query(t[a].l,t[b].l,x,l,mid);
	return res;
}

for(int i=1;i<=n;i++) update(1,n,roots[i],roots[i-1],a[i]);
query(roots[L-1],roots[R],x,1,n);
\end{lstlisting}
\subsubsection{区间内距离p第k近的距离}
\begin{lstlisting}
struct node
{
	ll sum,l,r;
}t[maxn*32];
int cnt;
void update(ll l,ll r,ll &x,ll y,ll pos){
	t[++cnt]=t[y];t[cnt].sum++;x=cnt;//复制节点并且更新
	if(l==r) return ;
	int mid=(l+r)>>1;
	if(mid>=pos) update(l,mid,t[x].l,t[y].l,pos);
	else update(mid+1,r,t[x].r,t[y].r,pos);
}
int query(int a,int b,int x,int l,int r)
{
	if(l==r)
	{
		if(l==x)
			return 0;
		else
			return l;
	}
	int mid=(l+r)>>1;
	int xx=t[t[b].l].sum-t[t[a].l].sum;
	int yy=t[t[b].r].sum-t[t[a].r].sum;
	int res=0;
	if(yy&&x>mid)
		res=query(t[a].r,t[b].r,x,mid+1,r);
	if(xx&&!res)
		res=query(t[a].l,t[b].l,x,l,mid);
	return res;
}

for(int i=1;i<=n;i++) update(1,n,roots[i],roots[i-1],a[i]);
query(roots[L-1],roots[R],x,1,n);
\end{lstlisting}

\subsubsection{区间内小于等于x的最大值}
\begin{lstlisting}
struct node
{
	ll sum,l,r;
}t[maxn*32];
int cnt;
void update(ll l,ll r,ll &x,ll y,ll pos){
	t[++cnt]=t[y];t[cnt].sum++;x=cnt;//复制节点并且更新
	if(l==r) return ;
	int mid=(l+r)>>1;
	if(mid>=pos) update(l,mid,t[x].l,t[y].l,pos);
	else update(mid+1,r,t[x].r,t[y].r,pos);
}
int query(int a,int b,int x,int l,int r)
{
	if(l==r)
	{
		if(l==x)
			return 0;
		else
			return l;
	}
	int mid=(l+r)>>1;
	int xx=t[t[b].l].sum-t[t[a].l].sum;
	int yy=t[t[b].r].sum-t[t[a].r].sum;
	int res=0;
	if(yy&&x>mid)
		res=query(t[a].r,t[b].r,x,mid+1,r);
	if(xx&&!res)
		res=query(t[a].l,t[b].l,x,l,mid);
	return res;
}

for(int i=1;i<=n;i++) update(1,n,roots[i],roots[i-1],a[i]);
query(roots[L-1],roots[R],x,1,n);
\end{lstlisting}
\subsubsection{区间数的种类数}
\begin{lstlisting}
#include<bits/stdc++.h>
using namespace std;
const int N = 1e6+5;
int n,q,a[N],p[N];
 
int rt[N*40],ls[N*40],rs[N*40],sum[N*40],cnt=0;
void up(int pre,int& o,int l,int r,int pos,int val) {
	o=++cnt;
	ls[o]=ls[pre];
	rs[o]=rs[pre];
	sum[o]=sum[pre]+val;
	if(l==r) return ;
	int m=(l+r)/2;
	if(pos<=m) up(ls[pre],ls[o],l,m,pos,val);
	else up(rs[pre],rs[o],m+1,r,pos,val);
}
 
int qu(int o,int l,int r,int ql,int qr) {
    if( ql<=l && qr>=r )  return sum[o];
    int ans = 0,m = (l+r)/2;
    if(ql<=m) ans += qu(ls[o],l,m,ql,qr);
    if(qr>m) ans += qu(rs[o],m+1,r,ql,qr);
    return ans;
}
 
int main(){
    scanf("%d",&n);
    for(int i=1;i<=n;i++) {
        scanf("%d",&a[i]);
        if(!p[a[i]]) {
            up(rt[i-1],rt[i],1,n,i,1);
        }else {
            int tp;
            up(rt[i-1],tp,1,n,p[a[i]],-1);
            up(tp,rt[i],1,n,i,1);
        }
        p[a[i]] = i;
    }
    scanf("%d",&q);
    while(q--) {
        int l,r;
        scanf("%d%d",&l,&r);
        int ans = qu(rt[r],1,n,l,r);
        printf("%d\n",ans);
    }
	return 0;
}
\end{lstlisting}
\subsubsection{区间内未出现过的最小自然数}
\begin{lstlisting}
#include<bits/stdc++.h>
using namespace std;
#define ll long long
const int N = 2e5+5;
const int M = 1e9;
 
int rt[N*30],ls[N*30],rs[N*30],mn[N*30],cnt=0;
void up(int pre,int& o,int l,int r,int val,int pos) {
	o=++cnt;
	ls[o]=ls[pre];
	rs[o]=rs[pre];
	mn[o]=pos;
	if(l==r) return ;
	int m=(l+r)/2;
	if(val<=m) up(ls[pre],ls[o],l,m,val,pos);
	else up(rs[pre],rs[o],m+1,r,val,pos);
	mn[o]=min(mn[ls[o]],mn[rs[o]]);
}
 
int qu(int o,int l,int r,int pos) {
	if(l==r) return l;
	int m=(l+r)/2;
    if( mn[ls[o]]<pos ) return qu(ls[o],l,m,pos);
    return qu(rs[o],m+1,r,pos);
}
 
int a[N],n,q,l,r;
int main(){
    scanf("%d%d",&n,&q);
    for(int i=1;i<=n;i++) {
        scanf("%d",&a[i]);
        up(rt[i-1],rt[i],0,M,a[i],i);
    }
    while(q--) {
        scanf("%d%d",&l,&r);
        int ans = qu(rt[r],0,M,l);
        printf("%d\n",ans);
    }
	return 0;
}
\end{lstlisting}
\subsection{树链剖分}
\begin{lstlisting}
#pragma GCC optimize(2)
#include<bits/stdc++.h>
#define rep(i, a, b)    for(int i = (a); i <= (int)(b); ++i)
#define per(i, a, b)    for(int i = (a); i >= (int)(b); --i)
#define debug(x)    cerr << #x << ' ' << x << endl;
#define ls x<<1
#define rs x<<1|1
using namespace std;

typedef long long ll;
const int MAXN = 1e6 + 7;

int son[MAXN], fa[MAXN], dep[MAXN], siz[MAXN], top[MAXN], tid[MAXN], rnk[MAXN], w[MAXN];
vector<int> G[MAXN];
int n, m, s, cur = 0;

struct SegTree{
    struct Node{
        int l, r;
        ll lz, sum;
        int mid(){return (l+r)>>1;}
        int size(){return (r-l+1);}
    }s[MAXN<<2];
    inline void pushdown(int x){
        s[ls].lz += s[x].lz;
        s[ls].sum += s[x].lz * s[ls].size();
        s[rs].lz += s[x].lz;
        s[rs].sum += s[x].lz * s[rs].size();
        s[x].lz = 0;
    }
    inline void pushup(int x){
        s[x].sum = s[ls].sum + s[rs].sum;
    }
    inline void build(int x, int l, int r){
        s[x].l = l; s[x].r = r;
        if(l == r){
            s[x].lz = 0;
            s[x].sum = w[rnk[l]];
            return;
        }
        int mid = s[x].mid();
        build(ls, l, mid);
        build(rs, mid + 1, r);
        pushup(x);
    }
    inline ll query(int x, int l, int r){
        if(s[x].l == l && s[x].r == r)  return s[x].sum;
        pushdown(x);
        int mid = s[x].mid();
        if(r <= mid)    return query(ls, l, r);
        else if(l > mid)    return query(rs, l, r);
        else    return query(ls, l, mid) + query(rs, mid + 1, r);
    }
    inline void updata(int x, int l, int r, int v){
        if(s[x].l == l && s[x].r == r){
            s[x].lz += v;
            s[x].sum += 1LL * v * s[x].size();
            return;
        }
        pushdown(x);
        int mid = s[x].mid();
        if(r <= mid)    updata(ls, l, r, v);
        else if(l > mid)    updata(rs, l, r, v);
        else {
            updata(ls, l, mid, v);
            updata(rs, mid + 1, r, v);
        }
        pushup(x);
    }
}st;
void dfs1(int x, int f = 0){
    son[x] = -1;
    siz[x] = 1;
    dep[x] = dep[f] + 1;
    fa[x] = f;
    for(int u: G[x]){
        if(u == f)  continue;
        dfs1(u, x);
        siz[x] += siz[u];
        if(son[x] == -1 || siz[son[x]] < siz[u])    son[x] = u;
    }
}

void dfs2(int x, int t){
    top[x] = t;
    cur++;
    tid[x] = cur;
    rnk[cur] = x;
    if(son[x] == -1)    return;
    dfs2(son[x], t);
    for(int u: G[x]){
        if(u != son[x] && u != fa[x]) dfs2(u, u);
    }
}


//链上更新
inline void linkadd(int u, int v, int w){
    int fu = top[u], fv = top[v];
    while(fu != fv){
        if(dep[fu] >= dep[fv]){
            st.updata(1, tid[fu], tid[u], w);
            u = fa[fu];
        } else {
            st.updata(1, tid[fv], tid[v], w);
            v = fa[fv];
        }
        fu = top[u];
        fv = top[v];
    }
    if(tid[u] > tid[v]) swap(u, v);
    st.updata(1, tid[u], tid[v], w);
}

//链上求和查询
inline ll linkquery(int u, int v){
    int fu = top[u], fv = top[v];
    ll res = 0;
    while(fu != fv){
        if(dep[fu] >= dep[fv]){
            res += st.query(1, tid[fu], tid[u]);
            u = fa[fu];
        } else {
            res += st.query(1, tid[fv], tid[v]);
            v = fa[fv];
        }
        fu = top[u];
        fv = top[v];
    }
    if(tid[u] > tid[v]) swap(u, v);
    res += st.query(1, tid[u], tid[v]);
    return res;
}

//子树查询
inline ll subtreequery(int x){
    return st.query(1, tid[x], tid[x] + siz[x] - 1);
}

//子树更新
inline void subtreeadd(int x, int w){
    return st.updata(1, tid[x], tid[x] + siz[x] - 1, w);
}

//查询LAC
inline int lca(int u, int v){
    int fu = top[u], fv = top[v];
    while(fu != fv){
        if(dep[fu] >= dep[fv])  u = fa[fu];
        else    v = fa[fv];
        fu = top[u];
        fv = top[v];
    }
    if(dep[u] > dep[v]) swap(u, v);
    return u;
}
int main() {
    scanf("%d %d %d", &n, &m, &s);
    rep(i, 1, n)    scanf("%d", &w[i]);
    int u, v, w, op;
    rep(i, 1, n-1){
        scanf("%d %d", &u, &v);
        G[u].push_back(v);
        G[v].push_back(u);
    }
    dfs1(s);
    dfs2(s, s);
    st.build(1, 1, n);
    while(m--){
        scanf("%d %d", &op, &u);
        if(op == 1) {
            scanf("%d %d", &v, &w);
            linkadd(u, v, w);
        } else if(op == 2) {
            printf("%lld\n", linkquery(u, u));
        } else {
            printf("%lld\n", subtreequery(u));
        }
    }
    return 0;
}
\end{lstlisting}
\subsection{树状数组}
\subsubsection{二位偏序求矩形内点的个数}
离线算法,以南京网赛A为例
\begin{lstlisting}
#include<bits/stdc++.h>
using namespace std;
const int N = 1e6+100;
const int mod = 1e9+7;
typedef long long ll;
const int INF = 0x3f3f3f3f;
const ll llINF = 0x3f3f3f3f3f3f3f3f;
#define rep(i,a,b) for(int i=(a);i<=(b);i++)
#define fep(i,a,b) for(int i=(a);i>=(b);i--)
inline bool read(ll &num) {
    char in;bool IsN=false;
    in=getchar();
    if(in==EOF) return false;
    while(in!='-'&&(in<'0'||in>'9')) in=getchar();
    if(in=='-'){ IsN=true;num=0;}
    else num=in-'0';
    while(in=getchar(),in>='0'&&in<='9'){
            num*=10,num+=in-'0';
    } 
    if(IsN) num=-num;
    return true;
}
ll T,n,p,m,c[N];
int lowbit(ll x){
    return (x&(-x));
}
void add(ll x,ll v){
    for(;x<N;x+=lowbit(x)){
         c[x]+=v;
         //cout<<x<<' '<<v<<endl;
    }
}
ll query(ll x){
    ll ans=0;
    for(;x;x-=lowbit(x)){
        ans+=c[x];
    }
    return ans;
}
//BIT
struct point{
    ll x,y;
    int flag;
}pp[600000];
bool cmp(point a,point b){
    if(a.x==b.x){
        if(a.y==b.y){
           return a.flag<b.flag;
        }
        return a.y<b.y;
    }
    return a.x<b.x;
}

ll dig(ll x){
    ll ans=0;
    while(x){
        ans+=x%10;
        x/=10;
    }
    return ans;
}
ll cal(ll x,ll y){//计算(x,y)处的值
    x=x-n/2-1;
    y=y-n/2-1;
    ll t=max(abs(x),abs(y));
    if(x>=y) return n*n-4*t*t-2*t-x-y;
    else return n*n-4*t*t+2*t+x+y;
}
map<pair<ll,ll>,ll> mmp;
ll yy[N],id[N];
ll x__1[100001],y__1[100001],x__2[100001],y__2[100001];
int main(){
    read(T);
    while(T--){
        mmp.clear();
        read(n);read(m);read(p);
        memset(c,0,sizeof(c));
        ll x,y,x_1,y_1,x_2,y_2;
        rep(i,1,m){
            read(x);read(y);
            pp[i]={x,y,0};
        }
        rep(i,1,p){
            read(x_1);read(y_1);read(x_2);read(y_2);
            x__1[i]=x_1;y__1[i]=y_1;x__2[i]=x_2;y__2[i]=y_2;
            pp[++m]={x_1-1,y_1-1,1};
            pp[++m]={x_2,y_2,1};
            pp[++m]={x_1-1,y_2,1};
            pp[++m]={x_2,y_1-1,1};
        }
        rep(i,1,m) yy[i]=pp[i].y;
        sort(yy+1,yy+m+1);
        int siz=unique(yy+1,yy+m+1)-yy-1;
        sort(pp+1,pp+m+1,cmp);
        rep(i,1,m){
            id[i]=lower_bound(yy+1,yy+siz+1,pp[i].y)-yy;
        }
        //离散化
        rep(i,1,m){
            if(pp[i].flag==0){
                add(id[i],dig(cal(pp[i].x,pp[i].y)));
                //cout<<pp[i].x<<' '<<pp[i].y<<' '<<id[i]<<' '<<dig(cal(pp[i].x,pp[i].y))<<endl;
            }
            else{
                mmp[{pp[i].x,pp[i].y}]=query(id[i]);
                //cout<<pp[i].x<<' '<<pp[i].y<<' '<<query(id[i])<<endl;
            }
        }
        rep(i,1,p){
            //cout<<x__2[i]<<' '<<y__2[i]<<endl;
            printf("%lld\n",mmp[{x__2[i],y__2[i]}]-mmp[{x__1[i]-1,y__2[i]}]-mmp[{x__2[i],y__1[i]-1}]+mmp[{x__1[i]-1,y__1[i]-1}]);
        }
    }
    return 0;
}
\end{lstlisting}
\subsubsection{树状数组求区间最值}
\begin{lstlisting}
struct BIT{
    ll e[MAXN];
    int lowbit(int x){
        return x & -x;
    }
    void upd(int x){
        int lx;
        while(x <= n){
            e[x] = a[x];
            lx = lowbit(x);
            for(int I = 1; I < lx; I <<= 1) e[x] = max(e[x], e[x-I]);
            x += lowbit(x);
        }
    }
    ll query(int l, int r){
        ll ans = 0;
        while(r >= l){
            ans = max(a[r], ans);
            r--;
            while(r >= l + lowbit(r)){
                ans = max(e[r], ans);
                r -= lowbit(r);
            }
        }
        return ans;
    }
}bit;
\end{lstlisting}
\subsubsection{二维树状数组}
查询二维前缀和
\begin{lstlisting}
#include<iostream>
#include<string.h>
#include<algorithm>
#include<stdio.h>

using namespace std;
const int N = 1e3+100;
int c[N][N];
int n;
int lowbit(int x){
    return x&(-x);
}
void update(int x,int y,int k){
    for(int i=x;i<=n;i+=lowbit(i)){
        for(int j=y;j<=n;j+=lowbit(j)){
            c[i][j]+=k;
        }
    }
}
int query(int x,int y){
    int ans=0;
    for(int i=x;i;i-=lowbit(i)){
        for(int j=y;j;j-=lowbit(j)){
            ans+=c[i][j];
        }
    }
    return ans;
}
\end{lstlisting}
\section{字符串}
\subsection{序列自动机}
\begin{lstlisting}
int nxt[N][27];//距离i往后最近的字母j的位置
void init(char *s){
    int l=strlen(s);
    for(int i=0;i<26;i++) nxt[l][i]=INF;
    for(int i=l-1;i>=0;i--){
        for(int j=0;j<26;j++){
            nxt[i][j]=nxt[i+1][j];
        }
        nxt[i][s[i]-'a']=i;
    }
}
\end{lstlisting}
\subsection{KMP计算next函数}
\subsubsection{vector版}
\begin{lstlisting}
#include<bits/stdc++.h>

using namespace std;
const int N = 2e6+100;
int n,m;int a[N],b[N];
vector<int> res;
vector<int> cal(vector<int> a){
	int n=(int)a.size();
	vector<int> nxt(n);
	for(int i=1;i<n;i++){
		int j=nxt[i-1];
		while(j>0&&a[i]!=a[j]) j=nxt[j-1];
		if(a[i]==a[j]) j++;
		nxt[i]=j;
	}
	return nxt;
}
\end{lstlisting}
\subsubsection{KMP匹配过程}
\begin{lstlisting}
int kmp(){
    int i,j;
    i=j=0;
	while(i<n&&j<m){
        if(s[i]==t[j]){
            i++;j++;
        }
		else if(!j){
            i++;
        }
		else{
            j=nxt[j-1];
        }
	}
	if(j==m) return i-m+1;
	else return -1;
}
\end{lstlisting}
\subsection{Z-function / Exkmp}
\begin{lstlisting}
vector<int> z_function(string s) {
  int n = (int)s.length();
  vector<int> z(n);
  for (int i = 1, l = 0, r = 0; i < n; ++i) {
    if (i <= r) z[i] = min(r - i + 1, z[i - l]);
    while (i + z[i] < n && s[z[i]] == s[i + z[i]]) ++z[i];
    if (i + z[i] - 1 > r) l = i, r = i + z[i] - 1;
  }
  return z;
}
\end{lstlisting}
\subsection{Manacher}
\begin{lstlisting}
#include<bits/stdc++.h>

using namespace std;
const int N = 1e6+100;
int p[N];
string s;
int manacher(string s){
	string t="";
	t+='*';
	for(int i=0;i<(int)s.size();i++){
		t+=s[i];
		t+='*';
	}
	int ans=0;
	int pos=0;int maxxright=0;
	for(int i=0;i<(int)t.length();i++){
		p[i]=maxxright>i?min(p[2*pos-i],maxxright-i):1;//关键
		while(i-p[i]>=0&&i+p[i]<(int)t.length()&&t[i-p[i]]==t[i+p[i]]) p[i]++;
		if(i+p[i]-1>maxxright){
			maxxright=i+p[i]-1;
			pos=i;
		}
		ans=max(ans,p[i]);
	}
	return ans-1;
}
int main(){
	cin>>s;
	cout<<manacher(s)<<endl;
	return 0;
}
\end{lstlisting}
\subsection{后缀数组}
从1到n输出sa[i]
sa[i]代表排名为i的下标
ran[i]代表下标为i的排名
\begin{lstlisting}
string s;
int ran[N],tmp[N],sa[N];
int n,k;
bool cmp(int i,int j){
    if(ran[i]!=ran[j]) return ran[i]<ran[j];
    int ri=i+k<=n?ran[i+k]:-1;
    int rj=j+k<=n?ran[j+k]:-1;
    return ri<rj;
}
void construct_sa(string s,int *sa){
    n=(int)s.length();
    for(int i=0;i<=n;i++){
        sa[i]=i;
        ran[i]=i<n?s[i]:-1;
    }
    for(k=1;k<=n;k*=2){
        sort(sa,sa+n+1,cmp);
        tmp[sa[0]]=0;
        for(int i=1;i<=n;i++){
            tmp[sa[i]]=tmp[sa[i-1]]+(cmp(sa[i-1],sa[i])?1:0);
        }
        for(int i=0;i<=n;i++) ran[i]=tmp[i];
    }
}
\end{lstlisting}
\subsection{双哈希}
\begin{lstlisting}
#include<bits/stdc++.h>
 
using namespace std;
typedef long long ull;
typedef long long ll;
const int N = 1e6+100;
const int base1 = 233;
const int base2 = 2333;
const int mod1 = 1e9+9;
const int mod2 = 1e9+7;
ull hashes1[N],p1[N];
ull hashes2[N],p2[N];
string s;
ull gethash1(int l,int r){
	if(l==0) return hashes1[r]%mod1;
	return (hashes1[r]-(hashes1[l-1]%mod1*p1[r-l+1]%mod1)%mod1+mod1)%mod1;
}
ull gethash2(int l,int r){
	if(l==0) return hashes2[r]%mod2;
	return (hashes2[r]-(hashes2[l-1]%mod2*p2[r-l+1]%mod2)+mod2)%mod2;
}
int main(){
	int n;cin>>n>>s;
	hashes1[0]=s[0];p1[0]=1;
	hashes2[0]=s[0];p2[0]=1;
	for(int i=1;i<n;i++){
		hashes1[i]=(hashes1[i-1]*base1%mod1+(ull)s[i]%mod1)%mod1;
		hashes2[i]=(hashes2[i-1]*base2%mod2+(ull)s[i]%mod2)%mod2;
		p1[i]=(base1*p1[i-1])%mod1;
		p2[i]=(base2*p2[i-1])%mod2;
	}
}
\end{lstlisting}
\subsection{字典树}
\subsubsection{指针版}
\begin{lstlisting}
struct Tire {
	const int MAXN = 4e5 + 7;
    int tr[MAXN][26], tot = 0;
    int cnt[MAXN];

    void insert(string s ) {
        int cur = 0, sz = s.size();
        cnt[cur] ++;    //插入的字符串个数
        for(int i = 0; i < sz; ++i) {
            int to = s[i] - 'a';
            if(!tr[cur][to]) tr[cur][to] = ++tot;
            cur = tr[cur][to];
        }
        cnt[cur]++;     //当前节点所表示的字符串的出现次数 + 1
    }

    int get(string s) {
        int cur = 0, sz = s.size();
        for(int i = 0; i < sz; ++i) {
            int to = s[i] - 'a';
            if(!tr[cur][to])    return 0;
            cur = tr[cur][to];
        }
        return cnt[cur];
    }
}trie;

\end{lstlisting}
\subsubsection{数组版01字典树}
从数组a中找出一个数使得k和它xor最大
\begin{lstlisting}
#include<bits/stdc++.h>

using namespace std;
const int N = (1e5+100)*33;
int tree[N][2];int a[N];
bool vis[N];
int tot;
void insert(int x){
    int now=0;
    for(int i=31;i>=0;i--){
        int id=(x>>i)&1;
        if(!tree[now][id]) tree[now][id]=++tot;
        now=tree[now][id]; 
    }
}
int find(int x){
    int ans=0;
    int now=0;
    for(int i=31;i>=0;i--){
        int id=!((x>>i)&1);
        ans*=2;
        if(tree[now][id]){
            ans++;now=tree[now][id];
        }
        else now=tree[now][!id];
    }
    return ans;
}
int main(){
    int T,n,m,s,all;
    all=1;
    scanf("%d",&T);
    while(T--){
        tot=0;
        memset(tree,0,sizeof(tree));
        // memset(vis,0,sizeof(vis));
        scanf("%d %d",&n,&m);
        for(int i=1;i<=n;i++) scanf("%d",&a[i]);
        for(int i=1;i<=n;i++){
            insert(a[i]);
        }
        printf("Case #%d:\n",all++);
        while(m--){
            scanf("%d",&s);
            printf("%d\n",find(s)^s);
        }
    }
    return 0;
}
\end{lstlisting}
\subsection{AC自动机}
\begin{lstlisting}
namespace AC{
    const int MAXN = 1e6 + 7;
    //注意字符集大小
    int tr[MAXN][30], tot = 0;
    int cnt[MAXN], fail[MAXN];
    void insert(string s){
        int cur = 0;
        for(int i = 0; i < s.size(); ++i){
            int to = s[i] - 'a';
            if(tr[cur][to] == 0)    tr[cur][to] = ++tot;
            cur = tr[cur][to];
        }
        cnt[cur]++;
    }
    queue<int> que;
    void build(){    
        for(int i = 0; i < 26; ++i){
            if(tr[0][i])  que.push(tr[0][i]);
        }
        while(!que.empty()){
            int cur = que.front();
            que.pop();
            for(int i = 0; i < 26; ++i){
                if(tr[cur][i]){
                    fail[tr[cur][i]] = tr[fail[cur]][i];
                    que.push(tr[cur][i]);
                } else tr[cur][i] = tr[fail[cur]][i];
            }
        }
    }
    //注意每次匹配结束 cnt 数组发生改变
    //如果需要再次匹配 需要重新建树
    int query(string s){
        int cur = 0, res = 0;
        for(int i = 0; i < s.size(); ++i){
            int to = s[i] - 'a';
            cur = tr[cur][to];
            for(int j = cur; j && cnt[j] != -1; j = fail[j]){
                res += cnt[j];
                cnt[j] = -1;
            }
        }
        return res;
    }
}

\end{lstlisting}
\subsection{最小表示法}
\begin{lstlisting}
int k = 0, i = 0, j = 1;
while (k < n && i < n && j < n) {
  if (sec[(i + k) % n] == sec[(j + k) % n]) {
    k++;
  } else {
    //最大表示法只需要将此处的 > 更改为 < 即可
    sec[(i + k) % n] > sec[(j + k) % n] ? i = i + k + 1 : j = j + k + 1;
    if (i == j) i++;
    k = 0;
  }
}
i = min(i, j);
\end{lstlisting}
\section{计算几何}
\subsection{基本的定义}
\begin{lstlisting}
#include<bits/stdc++.h>

using namespace std;
#define eps = (1e-10);
//点
struct Point{
    double x,y;
    Point(double _x=0,double _y=0):x(_x),y(_y){}
    Point operator + (Point p){return Point(x+p.x,y+p.y);}
    Point operator - (Point p){return Point(x-p.x,y-p.y);}
    Point operator * (double a){return Point(a*x,a*y);}
    Point operator / (double a){return Point(x/a,y/a);}
    double norm(){return x*x+y*y;}//模的平方
    double ABS() {return sqrt(norm());}//模
};
//线段
struct Segment{
    Point p1,p2;
};
//点积
double dot(Point a,Point b){
    return a.x*b.x+a.y*b.y;
}
//叉积
double cross(Point a,Point b){
    return a.x*b.y-a.y*b.x;
}
\end{lstlisting}
\subsection{点和线}
\subsubsection{计算投影的坐标}
\begin{lstlisting}
//p在S上的投影坐标
Point Projection(Point p,Segment s){
    Point alp=p-s.p1;Point beta=s.p2-s.p1;
    double res=dot(alp,beta)/beta.norm();
    return s.p1+(beta*res);
}
\end{lstlisting}
\subsubsection{计算p关于线段s的对称点}
\begin{lstlisting}
Point Reflection(Point p,Segment s){
    Point p1=s.p1;Point p2=s.p2;
    double A=(p1.y-p2.y);
    double B=-(p1.x-p2.x);
    double C=(p1.x-p2.x)*p1.y-p1.x*A;
    return {((B*B-A*A)*p.x-2*A*B*p.y-2*A*C)/(A*A+B*B),((A*A-B*B)*p.y-2*A*B*p.x-2*B*C)/(A*A+B*B)};
}
\end{lstlisting}
\subsubsection{判断线段是否相交}
\begin{lstlisting}
bool Intersect(Segment A ,Segment B){
    if(max(B.p1.x,B.p2.x)<min(A.p1.x,A.p2.x)) return 0;
    if(max(B.p1.y,B.p2.y)<min(A.p1.y,A.p2.y)) return 0;
    if(max(A.p1.y,A.p2.y)<min(B.p1.y,B.p2.y)) return 0;
    if(max(A.p1.x,A.p2.x)<min(B.p1.x,B.p2.x)) return 0;
    if(cross(A.p1-B.p1,A.p2-B.p1)*cross(A.p1-B.p2,A.p2-B.p2)>0) return 0;
    if(cross(B.p1-A.p1,B.p2-A.p1)*cross(B.p1-A.p2,B.p2-A.p2)>0) return 0;
    return 1;
}
\end{lstlisting}
\subsubsection{线段平行和正交判断}
\begin{lstlisting}
bool Parallel(Segment a,Segment b){
    Point alp=a.p2-a.p1;
    Point beta=b.p2-b.p1;
    if(cross(alp,beta)==0){
        return 1;
    }
    return 0;
}
//判断线段正交
bool Orthogonal(Segment a,Segment b){
    Point alp=a.p2-a.p1;
    Point beta=b.p2-b.p1;
    if(dot(alp,beta)==0){
        return 1;
    }
    return 0;   
}
\end{lstlisting}
\subsubsection{计算线段的交点}
\begin{lstlisting}
Point IntersectPoint(Segment A ,Segment B){
    double a=A.p1.y-A.p2.y;
    double b=A.p2.x-A.p1.x;
    double e=A.p1.x*(A.p1.y-A.p2.y)-A.p1.y*(A.p1.x-A.p2.x);
    double c=B.p1.y-B.p2.y;
    double d=B.p2.x-B.p1.x;
    double f=B.p1.x*(B.p1.y-B.p2.y)-B.p1.y*(B.p1.x-B.p2.x);
    double ansx=(e*d-f*b)/(a*d-c*b);
    double ansy=(a*f-c*e)/(a*d-c*b);
    if(ansx>=0) ansx=fabs(ansx);
    if(ansy>=0) ansy=fabs(ansy);
    return {ansx,ansy};  
}
\end{lstlisting}
\subsubsection{点到直线的距离}
\begin{lstlisting}
//点到直线的距离
double PLDis(Point a,Segment s){
    double A=s.p1.y-s.p2.y;
    double B=s.p2.x-s.p1.x;
    double C=(s.p1.x-s.p2.x)*s.p1.y-(s.p1.y-s.p2.y)*s.p1.x;
    return fabs(A*a.x+B*a.y+C)/sqrt((A*A+B*B));
}
\end{lstlisting}
\subsubsection{点到线段的最近距离}
\begin{lstlisting}
double Segdis(Point A ,Segment B){
    Point alp1=A-B.p1;
    Point beta1=B.p2-B.p1;
    Point alp2=A-B.p2;
    Point beta2=B.p1-B.p2;
    if(dot(alp1,beta1)<0||dot(alp2,beta2)<0){
        return min(Pointdis(A,B.p1),Pointdis(A,B.p2));
    }
    double a=B.p1.y-B.p2.y;
    double b=B.p2.x-B.p1.x;
    double c=B.p1.y*(B.p1.x-B.p2.x)-B.p1.x*(B.p1.y-B.p2.y);
    return fabs(a*A.x+b*A.y+c)/sqrt(a*a+b*b);
}
\end{lstlisting}
\subsection{多边形}
\subsubsection{多边形面积}
\begin{lstlisting}
vector<Point> polygon;
double Area(vector<Point> polygon){
    double ans=0;
    int n=(int)polygon.size();
    for(int i=0;i<n;i++){
        ans+=cross(polygon[i],polygon[(i+1)%n]);
    }
    return fabs(ans/2);
}
\end{lstlisting}
\subsubsection{判断多边形是否是凸包}
\begin{lstlisting}
bool Isconvex(vector<Point> polygon){
    int n=(int)polygon.size();
    polygon.push_back(polygon[0]);
    polygon.push_back(polygon[2]);
    for(int i=0;i<n;i++){
        Point a=polygon[i+1]-polygon[i];
        Point b=polygon[i+2]-polygon[i+1];
        if(cross(a,b)<0) return 0;
    }
    return 1;
}
\end{lstlisting}
\subsubsection{点和多边形的关系}
\begin{lstlisting}
//1代表在多边形上,2代表在内部,0代表在外部
int Contain(vector<Point> G,Point p){
    int n = G.size();
    bool x=0;
    for(int i=0;i<n;i++){
        Point a=G[i]-p,b=G[(i+1)%n]-p;
        if(abs(cross(a,b))<eps && dot(a,b)<eps) return 1;
        if(a.y>b.y) swap(a,b);
        if(a.y<eps&&eps<b.y&&cross(a,b)>eps) x=!x;
    }
    return (x?2:0);
}
\end{lstlisting}
\subsubsection{计算点集中的凸包}
\begin{lstlisting}
vector<Point> Andrew(vector<Point> G){
    sort(G.begin(),G.end(),cmp);
    vector<Point> up,down;
    int n=(int)G.size();
    up.push_back(G[0]);
    up.push_back(G[1]);
    down.push_back(G[n-1]);
    down.push_back(G[n-2]);
    for(int i=2;i<n;i++){
        while(up.size()>1&&cross(up[up.size()-2]-up[up.size()-1],G[i]-up[up.size()-1])<0){
             up.pop_back();
        }
        up.push_back(G[i]);
    }
    for(int i=n-3;i>=0;i--){
         while(down.size()>1&&cross(down[down.size()-2]-down[down.size()-1],G[i]-down[down.size()-1])<0){
              down.pop_back();
         }
         down.push_back(G[i]);       
    }
    vector<Point> ans;
    for(int i=down.size()-1;i>=1;i--) ans.push_back(down[i]);
    for(int i=up.size()-1;i>=1;i--) ans.push_back(up[i]);
    return ans;
}
\end{lstlisting}
\subsubsection{直线和圆的交点}
\begin{lstlisting}
vector<Point> CCL(Segment s,Point o,double r){
    vector<Point> res;
    Point x=Projection(o,s);
    double dis=PLDis(o,s);
    if(dis>r){//距离>r没有交点
        return res;
    }
    if(dis==r){//只有一个交点
        res.push_back(x);
        res.push_back(x);
    }
    double beta=sqrt(r*r-dis*dis);//勾股定理
    Point pp=s.p2-s.p1;
    pp=pp/pp.ABS();//单位向量
    Point ans1=x-pp*beta;
    Point ans2=x+pp*beta;
    res.push_back(ans1);
    res.push_back(ans2);
    return res;
}
\end{lstlisting}
\subsection{旋转卡壳}
\subsubsection{计算凸包直径}
\begin{lstlisting}
//旋转卡壳计算凸包直径
double Diameter(vector<Point> G){
    double ans=0;
    int n=G.size();
    for(int i=0,k=0;i<n;i++){
        while((G[i]-G[k]).norm()<(G[i]-G[(k+1)%n]).norm()) k=(k+1)%n;
        ans=max(ans,(G[i]-G[k]).ABS());
    }
    return ans;
}
\end{lstlisting}
\section{C++ pbds}
\subsection{头文件}
\begin{lstlisting}
#include<ext/pb_ds/assoc_container.hpp>
#include<ext/pb_ds/tree_policy.hpp>//用tree
#include<ext/pb_ds/hash_policy.hpp>//用hash
#include<ext/pb_ds/trie_policy.hpp>//用trie
#include<ext/pb_ds/priority_queue.hpp>//用priority_queue
using namespace __gnu_pbds;
\end{lstlisting}
\subsection{Hash}
\begin{lstlisting}
cc_hash_table<int,bool> h;  //拉链法
gp_hash_table<int,bool> h;  //探测法 稍快
\end{lstlisting}
\subsection{Tree}
\begin{lstlisting}
#define pii pair<int,int>
#define mp(x,y) make_pair(x,y)
tree<pii,null_type,less<pii>,rb_tree_tag,tree_order_statistics_node_update> tr;
pii //存储的类型
null_type //无映射(低版本g++为null_mapped_type)
less<pii> //从小到大排序
rb_tree_tag //红黑树
tree_order_statistics_node_update //更新方式
tr.insert(mp(x,y)); //插入;
tr.erase(mp(x,y)); //删除;
tr.order_of_key(pii(x,y)); //求排名 从0开始
tr.find_by_order(x); //找k小值,返回迭代器 从0开始
tr.join(b); //将b并入tr,前提是两棵树类型一样且没有重复元素
tr.split(v,b); //分裂,key小于等于v的元素属于tr,其余的属于b
tr.lower_bound(x); //返回第一个大于等于x的元素的迭代器
tr.upper_bound(x); //返回第一个大于x的元素的迭代器
//元素不能重复
//以上所有操作的时间复杂度均为O(logn)
\end{lstlisting}
\subsection{Trie}
\begin{lstlisting}
typedef trie<string,null_type,trie_string_access_traits<>,pat_trie_tag,trie_prefix_search_node_update> tr;
//第一个参数必须为字符串类型,tag也有别的tag,但pat最快,与tree相同,node_update支持自定义
tr.insert(s); //插入s
tr.erase(s); //删除s
tr.join(b); //将b并入tr
pair//pair的使用如下:
pair<tr::iterator,tr::iterator> range=base.prefix_range(x);
for(tr::iterator it=range.first;it!=range.second;it++)
    cout<<*it<<' '<<endl;
//pair中第一个是起始迭代器,第二个是终止迭代器,遍历过去就可以找到所有字符串了。
\end{lstlisting}

\subsection{优先队列}
\begin{lstlisting}
priority_queue<int,greater<int>,TAG> Q;//小根堆,大根堆写less<int>
/*其中的TAG为类型,有以下几种:
pairing_heap_tag
thin_heap_tag
binomial_heap_tag
rc_binomial_heap_tag
binary_heap_tag
其中pairing_help_tag最快*/
Q.push(x);
Q.pop();
Q.top();
Q.join(b);
Q.empty();
Q.size();
Q.modify(it,6);
Q.erase(it);
//以上操作我都不讲了,pbds里的优先队列还可以用迭代器遍历
\end{lstlisting}
%==============================正文部分==============================%
\end{document}
